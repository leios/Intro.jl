%%%%%%%%%%%%%%%%%%%%%%%%%%%%%%%%%%%%%%%%%
% Beamer Presentation
% LaTeX Template
% Version 1.0 (10/11/12)
%
% This template has been downloaded from:
% http://www.LaTeXTemplates.com
%
% License:
% CC BY-NC-SA 3.0 (http://creativecommons.org/licenses/by-nc-sa/3.0/)
%
% Modified by Jeremie Gillet in November 2015 to make an OIST Skill Pill template
%
%%%%%%%%%%%%%%%%%%%%%%%%%%%%%%%%%%%%%%%%%

%-------------------------------------------------------------------------------
%	PACKAGES AND THEMES
%-------------------------------------------------------------------------------

\documentclass{beamer}

\mode<presentation> {

\usetheme{Madrid}

\definecolor{OISTcolor}{rgb}{0.65,0.16,0.16}
\usecolortheme[named=OISTcolor]{structure}

%\setbeamertemplate{footline} % To remove the footer line in all slides uncomment this line
%\setbeamertemplate{footline}[page number] % To replace the footer line in all slides with a simple slide count uncomment this line

\setbeamertemplate{navigation symbols}{} % To remove the navigation symbols from the bottom of all slides uncomment this line
}
\setbeamertemplate{itemize items}[triangle]
\setbeamertemplate{enumerate items}[default]

\usepackage{graphicx} % Allows including images
\usepackage{booktabs} % Allows the use of \toprule, \midrule and \bottomrule in tables
\usepackage{textpos} % Use for positioning the Skill Pill logo
\usepackage{fancyvrb}
\usepackage{tikz}
\usepackage{hyperref}
\usepackage{listings}

\definecolor{dkgreen}{rgb}{0,0.6,0}
\definecolor{gray}{rgb}{0.5,0.5,0.5}
\definecolor{mauve}{rgb}{0.58,0,0.82}

\lstset{frame=tb,
  language=python,
  aboveskip=3mm,
  belowskip=3mm,
  showstringspaces=false,
  columns=flexible,
  basicstyle={\small\ttfamily},
  numbers=none,
  numberstyle=\tiny\color{gray},
  keywordstyle=\color{blue},
  commentstyle=\color{dkgreen},
  stringstyle=\color{mauve},
  breaklines=true,
  breakatwhitespace=true,
  tabsize=3
}

%-------------------------------------------------------------------------------
%	TITLE PAGE
%-------------------------------------------------------------------------------

\title[Intro.jl]{Introduction to Julia} % The short title appears at the bottom of every slide, the full title is only on the title page
\subtitle{Session 3: Bits and Bobs}

\author{James Schloss} % Your name
\institute[LeiosLabs] % Your institution as it will appear on the bottom of every slide, may be shorthand to save space
{
\textit{jrs.schloss@gmail.com} % Your email address
}
\date{October 9, 2023} % Date, can be changed to a custom date

\begin{document}

\begin{frame}
\vspace*{1.4cm}
\titlepage % Print the title page as the first slide
\end{frame}

\setbeamertemplate{background}{} % No background logo after title frame

\begin{frame}
\frametitle{Last time on Intro.jl}
\center \Huge{Quick Recap}
\end{frame}

\begin{frame}
\frametitle{Performance and Debugging}
\begin{itemize}
\item @time
\item @benchmark
\item @code\_lowered, @code\_typed
\end{itemize}
\end{frame}

\begin{frame}
\frametitle{Ok, let's wrap up!}
\begin{itemize}
\item Add all necessary packages to the NewPackage.jl src file
\item Create the github url and discuss github utilities
\end{itemize}
\end{frame}

\begin{frame}
\frametitle{PackageCompiler}
\begin{itemize}
\item Sysimages
\item Apps
\item Libraries
\end{itemize}
\pause
\center We'll skip libraries for now...
\end{frame}

\begin{frame}
\frametitle{Calling other languages}
\begin{itemize}
\item Python
\item C / Fortran
\item C++
\end{itemize}
\end{frame}

\begin{frame}[fragile]
\frametitle{Python: calling libraries}
\begin{lstlisting}
using PyCall
math = pyimport("math")
math.sqrt(5)
\end{lstlisting}

This works for any library:

\begin{lstlisting}
sudo pacman -S python-numpy
julia> @pyimport numpy
julia> arr = numpy.array([[1, 2, 3], [4, 5, 6]], numpy.int32)
2x3 Matrix{Int32}:
 1  2  3
 4  5  6

julia> typeof(arr)
Matrix{Int32} (alias for Array{Int32, 2})
\end{lstlisting}
\end{frame}

\begin{frame}[fragile]
\frametitle{Python: calling local code}
\begin{lstlisting}
julia> py"print('Hello World!')"
Hello World!

julia> py"""
       import numpy as np

       def sinpi(x):
           return np.sin(np.pi * x)
       """

julia> py"sinpi"(1)

julia> PyCall.@pyinclude("hw.py")
Hello world!

\end{lstlisting}
\end{frame}

\begin{frame}[fragile]
\frametitle{Python: calling local code}

hw.py:
\begin{lstlisting}
def check():
    print("Hello world!")
\end{lstlisting}

\begin{lstlisting}
julia> py"check()"
Hello world!
\end{lstlisting}
\end{frame}

\begin{frame}[fragile]
\frametitle{C / Fortran: calling libraries}
\begin{lstlisting}
julia> @ccall clock()::Int32
11750209
\end{lstlisting}
\end{frame}

\begin{frame}[fragile]
\frametitle{C / Fortran: calling local code}
myclib.h:

\begin{lstlisting}
extern int get2 ();
extern double sumMyArgs (float i, float j);
\end{lstlisting}

myclib.c:

\begin{lstlisting}
int get2 (){
 return 2;
}
double sumMyArgs (float i, float j){
 return i+j;
}
\end{lstlisting}
Note that we need to define the function we want to use in Julia as extern in the C header file.

\end{frame}

\begin{frame}[fragile]
\frametitle{C / Fortran: calling local code}

We can then compile the shared library with gcc, a C compiler:

\begin{lstlisting}
gcc -o myclib.o -c myclib.c
gcc -shared -o libmyclib.so myclib.o -lm -fPIC
\end{lstlisting}
We are now ready to use the C library in Julia:

\begin{lstlisting}
const myclib = joinpath(@__DIR__, "libmyclib.so")
a = ccall((:get2,myclib), Int32, ())
b = ccall((:sumMyArgs,myclib), Float64, (Float32,Float32), 2.5, 1.5)
\end{lstlisting}

Remember that you can wrap these functions in Julia:

\begin{lstlisting}
function sumMyArgs(a::Flaot32, b::Float32)
    return ccall((:sumMyArgs,myclib), Float64, (Float32,Float32), a, b)
end
\end{lstlisting}

\end{frame}

\begin{frame}[fragile]
\frametitle{C / Fortran: calling local code}
simplemodule.f95

\begin{lstlisting}
module simpleModule

contains
function foo(x)
  integer :: foo, x
  foo = x * 2
end function foo

end module simplemodule
\end{lstlisting}

Which has to be compiled with

\begin{lstlisting}
gfortran simplemodule.f95 -o simplemodule.so -shared -fPIC
\end{lstlisting}

producing the simplemodule.so shared library file.

\end{frame}

\begin{frame}[fragile]
\frametitle{C / Fortran: calling local code}
Then, in Julia,

\begin{lstlisting}
a = Int32[3]
ccall((:__simplemodule_MOD_foo, "./simplemodule.so"), Int32, (Ptr{Int32},), a)
\end{lstlisting}
\end{frame}

\begin{frame}[fragile]
\frametitle{C++: It's messy...}

There are a few packages that ``work'' for C++:
\begin{itemize}
\item Cxx.jl: Directly embed C++ code in Julia (like PyCall)
\item CxxWrap: Wraps C++ code so you can export the functions and use them in base Julia
\item CxxInterface: An updated version of Cxx.jl
\end{itemize}
\end{frame}

\begin{frame}
\frametitle{PkgCompiler.jl}
Easy way to compile your packages
\end{frame}

\begin{frame}
\frametitle{Interacting with github}
\begin{itemize}
\item TagBot
\item CI
\end{itemize}
\end{frame}

\end{document}
